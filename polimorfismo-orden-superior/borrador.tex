\documentclass[spanish]{article}
\usepackage[spanish]{babel}
\usepackage[utf8x]{inputenc}

\title{
  Sistemas de Orden Superior \\
  Investigación \\
  CC3007 - Construcción de Compiladores
  \author{Carlos López Camey \\
  Universidad del Valle de Guatemala}
}
\date{Septiembre del 2010}
\maketitle

\begin{document}

\section{Introducción}

\section{Motivación}

Los sistemas de tipos estáticos han probado ser efectivos y un modelo práctico para diseñar y verificar especificaciones de una interfaz. No obstante, los primeros sistemas de tipos estáticos como el de \textbf{Pascal} están basados en la idea de que sus funciones y procedimientos, y por lo tanto sus operandos, tienen un único tipo (lenguajes monomórficos). En contraste, los lenguajes polimórficos permiten a valores o variables tener más de un tipo. 

Los sistemas polimórficos permiten hacer abstracciones sobre un tipo. \textbf{Ejemplo:} podemos decir que una lista de enteros (denotada \textit{List[Int]}) es un tipo donde \textit{List} es un constructor de tipos, que por ahora diremos que su función es ``tomar'' un tipo para revelar otro tipo.  

La abstracción sobre tipos es muy útil. Podríamos, por ejemplo, definir una función \textit{reverse} sobre una \textit{List[A]} que funciona para cualquier elemento de tipo \textit{A}. Decimos que los lenguajes que nos permiten hacer este tipo de abstracción tienen \textbf{Polimórfismo de primer-orden}; lenguajes como \textbf{Java} y \textbf{C\#} tienen polimorfismo de primer-orden.

Otros lenguajes de programación nos permiten definir nuestros propios operadores de tipos, es decir, abstraen sobre constructores de tipos. A esta característica le llamamos \textbf{Polimórfismo de alto-orden}.

\subsection{Objetivos}

\subsection{Objetivos teóricos:}
\begin{enumerate}
\item Definir qué es polimorfismo y los tipos de polimorfismo de primer-orden que existen.
\item Definir la equivalencia de tipos sobre el polimorfismo de primer-orden y de orden-superior.
\item Utilizando el isomorfismo de \textit{Curry-Howard}:
  \begin{itemize}
  \item Definir el sistema $F$ para el polimorfismo de primer-orden y resaltar propiedades más importantes.
  \item Definir el sistema $F_\omega$ para el polimorfismo de orden-superior y resaltar propiedades más importantes.
  \end{itemize}
\end{enumerate}
\pagebreak
\subsection{Objetivos prácticos:}
\begin{enumerate}
\item Borramiento y reconstrucción de tipos.
\item Verificación de tipos en un compilador.
\item Ejemplificar con un lenguaje de programación que soporte polimorfismo de orden-superior.
  \begin{itemize}
  \item Definición y usos de un \textit{Monad}.
  \end{itemize}
\end{enumerate}

\section{Polimorfismo de primer orden}

\subsection{Tipos de polimorfismo}

\subsection{El sistema $F$}

\subsection{\textit{Erasure} y Reconstrucción de tipos}

\subsection{Ejemplos}

\section{Polimorfismo de alto orden}

\subsection{El sistema $F_\omega$}

\subsection{Ejemplos}

\section{Verificación de tipos}

\section{Bibliografía}

\end{document}


