\documentclass[12pt,letterpaper]{article}
\usepackage{listings}
\usepackage[usenames,dvipsnames]{color}
\usepackage{ifpdf}
\usepackage{verbatim}

\author{Carlos L\'{o}pez}
\title{Semántica de KDecaf}

% "define" KDecaf
\begin{comment}
\lstdefinelanguage{Decaf}{
  morekeywords={class,struct,true,false,void,%
    if,else,while,return,int,char,boolean},
  otherkeywords={{,},;,+,-,/,*,<=,>,<,<=,>=,!=,==,\%,@},
  sensitive=true,
  morecomment=[l]{//},
  morecomment=[n]{/*}{*/},
  morestring=[b]",
  morestring=[b]',
  morestring=[b]"""
}
\end{comment}

\lstdefinelanguage{Scala}{
  morekeywords={abstract,case,catch,class,def,do,else,extends,
  false,final,finally,for,if,implicit,import,match,mixin,
  new,null,object,override,package,
  private,protected,requires,return,sealed,
  super,this,throw,trait,true,try,type,val,var,while,with,yield},
  otherkeywords={{,},;,+,-,~>,<~,~,|,/,*,<=,>,<,<=,>=,!=,==,\%,@},
  sensitive=true,
  morecomment=[l]{//},
  morecomment=[n]{/*}{*/},
  morestring=[b]",
  morestring=[b]',
  morestring=[b]"""
}

\usepackage{color}
\definecolor{dkgreen}{rgb}{0,0.6,0}
\definecolor{gray}{rgb}{0.5,0.5,0.5}
\definecolor{mauve}{rgb}{0.58,0,0.82}
 
% Default settings for code listings
\lstset{frame=tb,
  language=Scala,
  aboveskip=3mm,
  belowskip=3mm,
  showstringspaces=false,
  columns=flexible,
  basicstyle={\small\ttfamily},
  numbers=none,
  numberstyle=\tiny\color{gray},
  keywordstyle=\color{blue},
  commentstyle=\color{dkgreen},
  stringstyle=\color{mauve},
  frame=none,
  breaklines=true,
  breakatwhitespace=true
  tabsize=3
}

\lstset{frame=tb,
  language=Scala,
  aboveskip=3mm,
  belowskip=3mm,
  showstringspaces=false,
  columns=flexible,
  basicstyle={\small\ttfamily},
  numbers=none,
  numberstyle=\tiny\color{gray},
  keywordstyle=\color{blue},
  commentstyle=\color{dkgreen},
  stringstyle=\color{mauve},
  frame=none,
  breaklines=true,
  breakatwhitespace=true
  tabsize=3
}


\begin{document}
\maketitle

\section{Antecedentes}

El $stream$ de caracteres de entrada es mapeado por el parser \textbf{KDecafParser} a instancias de nodos del AST; este se encarga de construír nodos como \textbf{Program} o un nodo \textbf{Declaration} que puede ser ya sea \textbf{VarDeclaration}, \textbf{StructDeclaration} o \textbf{MethodDeclaration}, n\'{o}tese que \textbf{Declaration} es un nodo abstracto, es decir, el parser no podría construír un \textbf{Declaration} sin que este fuera una instancia de sus tipos espec\'{i}ficos.

\section{Estructuras}

Existe un trait (similar a una interfaz en Java), que extiende de \textbf{$Scope => SemanticResult$} en Scala, eso significa que es una funci\'{o}n que toma un argumento de tipo \textbf{Scope} y devuelve un valor de tipo \textbf{SemanticResult}.

Estos traits y clases se ecuentran definido en un archivo Semantics.scala:

\lstinputlisting{/home/kmels/code/uvg/cc3007/src/main/scala/compiler/Semantics.scala}

Nótese que un resultado semántico puede ser cualquier instancia de los siguientes casos:

\begin{itemize}
  \item Una lista de resultados semánticos: \textbf{SemanticResults}
  \item Un resultado semántico que tiene sentido: \textbf{SemanticSuccess}
  \item Un error semántico: \textbf{SemanticError}
\end{itemize}

Cada nodo implementa a \textbf{SemanticRule} por lo que se podr\'{i}a decir que cada nodo tambi\'{e}n es una regla sem\'{a}ntica. Los resultados de las aplicaciones de las reglas semánticas son evaluados cuando se necesita, por ejemplo \textbf{Program} necesita de los resultados semánticos de cada una de sus declaraciones. De esta manera, los resultados semánticos se pueden propagar.

Todos los nodos tienen un tipo interno, esto lo restringe el trait InnerType, que forza a cada nodo a tener una funci\'{o}n que no recibe argumentos y devuelve el tipo del nodo.

\begin{lstlisting}
trait Node extends Positional with InnerType{
  override def toString = getClass.getName
}

trait InnerType {
  val getUnderlyingType: () => String
}
\end{lstlisting}

Algunos nodos no necesitan calcular su valor siempre que les sea requerido, esto se puede abstraer construyendo otros traits que extendien a \textbf{InnerType} y aplicandolos en nodos de la siguiente forma:

\begin{lstlisting}
trait InnerInt extends InnerType{
  val getUnderlyingType = () => "Int"
}

case class ExpressionAdd(val exp1:Expression,val exp2:Expression)(implicit val m:Manifest[Int]) extends BinaryOperation[Int] with InnerInt

case class ExpressionSub(val exp1:Expression,val exp2:Expression)(implicit val m:Manifest[Int]) extends BinaryOperation[Int] with InnerInt
case class ExpressionMult(val exp1:Expression, val exp2:Expression)(implicit val m:Manifest[Int]) extends BinaryOperation[Int] with InnerInt
\end{lstlisting}

\section*{Ejemplo de Regla Sem\'{a}ntica}

Un nodo \textbf{Assignment} que representa a la producci\'{o}n \textbf{$Location$ '=' $Expression$} verifica lo siguiente:
\begin{itemize}
  \item El tipo de Location es el mismo que el de expression.
  \item Location exista como variable declarada.
\end{itemize}

\begin{lstlisting}
case class Assignment(val location:Location,val expression:Expression) extends Statement with InnerTypeVoid{

  val semanticAction = SemanticAction(
    attributes => {
      val typeEqualityResult = typesShouldBeEqualIn(
	location,expression,
	"cannot assign expression of type "+expression.getUnderlyingType()+" to "+location.name+" of declared type "+location.getUnderlyingType()
      )

      SemanticResults(typeEqualityResult,location.semanticAction(attributes),expression.semanticAction(attributes))
    }
  )
}
\end{lstlisting}

Para m\'{a}s informaci\'{o}n sobre las reglas sem\'{a}nticas y nodos existentes, visitar \cite{kmels-rep-cc3007}:

\begin{thebibliography}{1}
%referencias anteriores
\bibitem{kmels-rep-cc3007} Repositorio Construcci\'{o}n de Compiladores, kmels.net, \url{http://uvg.kmels.net/code/cc3007}.
\end{thebibliography}
\end{document}
